\newpage
\section{Realisering}
\label{realiseringOgTest}

\subsection{Hvit støy}
\label{hvitStoeoy}
For å realisere LFSRen ble det benyttet en FPGA av typen Latrice ice40, og lagd et shift register med 32 D-vipper. inngangen til den føreste D-vippen ble koblet til XOR porter som hadde inngangene koblet til utgangen av D-vippe nummer 1, 5, 6 og 31 som ambefalt av Max Maxfield \cite{LFSR}. Med denne konfigurasjonen vil signalet gjenta seg selv etter ca 4.3 milliarder klokkesykluser, noe som vil være godt innenfor dette prosjektes designkrav om et pseudo tilfeldig signal. For å iverksette LFSRen ble en knapp koblet opp til ingangen til den første D-vippen ved hjelp av en OR port. Diagramet for LFSRen kan sees i \autoref{fig:LFSRdiagram}
% og koden kan sees i vedlegg \ref{LFSRkode}.

\begin{figure} [!h]
\centering
\includegraphics[width=1\linewidth]{Bilder/LFSR.png}
\caption{LFSR}
\label{fig:LFSRdiagram}
\end{figure}

\subsection{Aktivt filter}
\label{aktivtFilterRealisering}
Dette designet er tiltenkt fungere som et båndpassfilter med tiltenkt senterfrekvens på 3000Hz. Det ble regnet ut verdier ved å benytte likninge som beskrevet i \autoref{aktivtFilter} og disse verdiene ble brukt til å sette opp kretsen i \autoref{fig:SallenKey}. Verdiene som ble benyttet kan ses i \autoref{tab:filterVerdier}. For å realisere dette som et båndpassfilter ble Delyiannis-Friend topologien som nevnt tidligere realisert som vist i \autoref{fig:realisertFilter}. Filteret har en Q-verdi på 10.2 og en båndbredde på 98Hz som kan leses ut fra \autoref{fig:filterTest}.

\begin{table}[H]
    \centering
    \caption{Verdier for realisering av båndpassfilteret.}
    \label{tab:filterVerdier}
    \begin{tabular}{|c|c|c|c|c|c|}
    \hline
    Komponent & $R_1$ & $R_2$ & $R_3$ & $C_{1}$ & $C_{2}$ \\ \hline
    Teoretisk verdi & \SI{53}{\kilo\ohm} & \SI{266}{\Omega} & \SI{100}{\kilo\ohm} & \SI{10}{\nano\farad} & \SI{10}{\nano\farad} \\ \hline
    Realisert verdi & \SI{53}{\kilo\ohm} & \SI{240}{\Omega} & \SI{99.9}{\kilo\ohm}  & \SI{10}{\nano\farad} & \SI{10}{\nano\farad} \\ \hline
    \end{tabular}
\end{table}  
\newpage



\begin{figure}[H]
    \centering
    \resizebox{0.6\textwidth}{!}{%
    \begin{circuitikz}
    \tikzstyle{every node}=[font=\LARGE]
    \draw 
     (6.75,15.25) to[R] (8.75,15.25) %R1

     (8.75,15.25) to[R] (8.75,13.25) %R2
     (8.75,13.25) to (8.75,13.25) node[ground]{}

     (8.75,15.25) to[C] (10.75,15.25) %C1

     (13,14.75) node[op amp,scale=1, yscale=-1 ] (opamp2) {}
     (opamp2.+) to[short] (11.5,15.25)
     (opamp2.-) to[short] (11.5,14.25)
     (14.2,14.75) to[short](14.5,14.75)
     
     [](11.5,14.25) to[short] (11.5,13.25)
     (11.5,13.25) to (11.5,13.25) node[ground]{}

     (8.75,17.75) to[C] (14,17.75) %C2
     [](14,14.75) to[short] (14,17.75)
     [](8.75,15.25) to[short] (8.75,17.75)

     (10.75,16.5) to[R] (14,16.5) %R3
     [](10.75,15.25) to[short] (10.75,16.5)
     

     [](10.75,15.25) to[short] (11.5,15.25)
     [](14.5,14.75) to[short, -o] (15,14.75)
     [](6.75,15.25) to[short, -o] (6.25,15.25)
     (8.75,15.25) to[short, -*] (8.75,15.25)
     (10.75,15.25) to[short, -*] (10.75,15.25)
     (14,14.75) to[short, -*] (14,14.75)
     ;
     \node [font=\large] at (15,14.25) {$V_o$};
     \node [font=\large] at (6.25,14.75) {$V_s$};
     \node [font=\large] at (7.75,15.75) {$R_1$};
     \node [font=\large] at (9.75,16) {$C_2$};
     \node [font=\large] at (11.25,18.5) {$C_1$};
     \node [font=\large] at (9.25,14.25) {$R_2$};
     \node [font=\large] at (12.5,17) {$R_3$};

    \end{circuitikz}
    }%
    \caption{Delyiannis-Friend båndpassfilter.}
    \label{fig:realisertFilter}
    \end{figure}

\subsection{Test og resultat}
\label{test}
For å teste om filteret fungerete ble det først gjort en network analyse på filteret i seg selv for å finne amplituderesponsen til systemet. Dette ble gjort ved å benytte et signalgenerator og en oscilloskop. Signalet som ble generert var en sinusbølge med frekvenser fra 0 til 10kHz. Testresultatet kan ses i \autoref{fig:filterTest}. Filteret hadde knekkfrekvensene 2943Hz og 3071Hz og ut ifra det kan vi se at bånbredden ble litt bredere enn det som ble beregnet tidligere med en målt båndbredde på 128Hz.

\begin{figure} [h]
\centering
\includegraphics[width=0.8\linewidth]{Bilder/Network2.png}
\caption{Test av filteret}
\label{fig:filterTest}
\end{figure}

Deretter ble systemet i sin helhet testet ved å koble systemene sammen. I \autoref{fig:systemTest} så kan man se signalet $V_s$ som er signalet som kommer ut av FPGAen og $V_{o}$ som er signalet som kommer ut av filteret. Det er tydelig at filteret fungerer som det skal og at det filtrerer bort frekvenser som ikke er i nærheten av senterfrekvensen. Amplituderesponsen til filteret slipper mest gjennom frekvenser rundt 3kHz som forventet men med en veldig redusert amplitude. Den er fremdeles hørbar uten å trenge å forsterkes hvis den spilles av direkte fra WaveForms \cite[Digilent Inc.]{Ocili}. Det var også hørbart med støy i bakrunnen som forventet, for å ungå dette kunne man brukt et høyere ordens filter og forsterket signalet etter filteret.



\begin{figure} [!h]
\centering
\includegraphics[width=0.8\linewidth]{Bilder/spectrum_Bondpass2.png}
\caption{Test av systemet}
\label{fig:systemTest}
\end{figure}

Et bilde som viser den oppkoblete kretsen kan ses i \autoref{fig:oppkobling}.

\begin{figure} [!h]
\centering
\includegraphics[width=0.6\linewidth]{Bilder/oppkobling.jpg}
\caption{Oppkobling av krets}
\label{fig:oppkobling}
\end{figure}
