\section{Problembeskrivelse}
\label{sec:innledning}

En utfordring i et system kan være at en signalkilde ikke klarer å levere nok strøm til en last. Det kan hende at spenningsnivået er høyt nok, men strømstyrken er ikke nok til å levere effekten lasten krever. En løsning på dette problemet er å bruke en buffer. Konseptet for et slikt system vises i figuren under.

\vspace{1cm}
\begin{figure}[!h]
    \centering
    \begin{circuitikz} [american voltages]
    \draw
    (0,1) to (0,0) node[ground]{}
    (0,2) to [V=$v_0$] (0,1)
    (0,2) to (0,3)
    
    (0,3) to [R, l^=$R_\text{K}$] (3,3)
    
    (3,3) to [short, -*] (4,3)
    (4,3.5) node[]{$v_1$}
    
    (4,3) to (5.5,3)
    (6,3) node[buffer]{}
    
    (6.5,3) to [short, -*] (7.5,3)
    (7.5,3.5) node[]{$v_2$}
    
    (7.5,3) to [short,-] (10,3)
    (10,3) to [R, l^=$R_\text{L}$] (10,0.5)
    
    (10,0.5) to (10,0) node[ground]{}
    ;
    \end{circuitikz}
    \caption{En bufferkrets med en kilde, buffer og last.}
    \label{fig:01}
\end{figure}
\vspace{1cm}

Systemet i figur \ref{fig:01} har en spenningskilde $v_0$, en utgangsmotstand $R_\text{K}$ for kilden, en last $R_\text{L}$ og en buffer med inngangssignal $v_1$ og utgangssignal $v_2$. Bufferen må være slik at

\begin{equation}
    v_2 \approx v_1 \approx v_0
\end{equation}

og den må være mest mulig uavhengig av $R_\text{K}$ og $R_\text{L}$ for å gi et presist resultat.